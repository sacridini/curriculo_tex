\documentclass{article}
\usepackage[brazilian]{babel} % PT-BR
\usepackage[utf8]{inputenc} % UTF-8
\usepackage[margin=1in]{geometry} % configura o espaçamento das margens do documento

\usepackage{titling}
\usepackage{titlesec} % para configuração das sections
\titleformat{\section}{\huge\bfseries}{}{0em}{}[\titlerule]
\titleformat{\subsection}{\bfseries\Large}{$\bullet$ }{0em}{}
\titleformat{\subsubsection}[runin]{\bfseries}{}{0em}{}[:] % "runin" serve para não indentar o texto.
\titlespacing{\subsubsection}{0em}{0.5em}{1em} % espaçamento config

% redefine o comando \maketitle
\renewcommand{\maketitle}{
\begin{center}{
\huge\bfseries
\theauthor
}
\vspace{.25em} \\
https://github.com/sacridini -- http://lattes.cnpq.br/0392869318849187

\end{center}
}

% MAIN
\begin{document}
\title{Curriculum Vitae}
\author{Eduardo Ribeiro Lacerda}
\maketitle

% ------------------------------------------------------ % 

\section{Informações Pessoais}
\subsubsection{Endereço}
Rua Humaitá, Humaitá - Rio de Janeiro
\subsubsection{Complemento}
Nº229, Apt. 105
\subsubsection{E-mail} 
eduardolacerdageo@gmail.com
\subsubsection{Contato}
(21) 97287-3254  \\

\subsubsection{Info}
Atualmente atuo como dicente do curso de doutorado em Geografia sob orientação do Prof. Dr. Raúl Sanchéz Vicens na Universidade Federal Fluminense. Realizo pesquisas na área de processamento de imagens em ambientes computacionais de alto desempenho em nuvem voltados tanto para a área de climatologia urbana assim como para a área de análises de tendências em séries temporais com o objetivo de estudar a expansão agrícola no centro-oeste brasileiro. A pesquisa consiste na utilização da plataforma (PaaS) Google Earth Engine (JavaScript/Python) e também no processamento local de imagens MODIS utilizando algoritmos desenvolvidos em R e C++. Também trabalho no programa de monitoramento de desmatamento do Estado do Rio de Janeiro ligado ao projeto Olho no Verde (SEA/INEA/UFRJ) e como analista de banco de dados espaciais no projeto Fundo Verde ligado à COPPE/UFRJ.

% ------------------------------------------------------ % 

\section{Formação Acadêmica}
\subsection{Doutorado}
\subsubsection{Geografia}
(2017 - Atual) Universidade Federal Fluminense - UFF (CAPES 6) 
\subsection{Mestrado}
\subsubsection{Geografia}
(2014 - 2016) Universidade Federal do Rio de Janeiro - UFRJ (CAPES 7)
\subsection{Graduação}
\subsubsection{Geografia}
(2009 - 2013) Universidade do Estado do Rio de Janeiro - UERJ
\subsubsection{Sistemas de Informação}
(2018 - Atual) Universidade Federal Fluminense - UFF
\subsubsection{Sistemas de Informação (Não concluído)}
(2010 - 2014) Universidade Estácio de Sá - UNISA
\subsubsection{Estatística (Não concluído)}
(2008 - 2009) Escola Nacional de Ciências Estatísticas - ENCE

% ------------------------------------------------------ % 

\section{Conhecimentos Técnicos}
\subsection{SIG/Sensoriamento Remoto}
ArcGIS, QGIS, eCognition, ENVI, ERDAS, IDRISI (TerrSet), Spring, OTB, GDAL, PCI Geomatica, Fusion, LAStools, Google Earth Engine
\subsection{Linguagens de Programação}
\subsubsection{Compiladas}
C/C++, Go
\subsubsection{Interpretadas}
R, Python, C\#, JavaScript, Shell Script
\subsubsection{Markup}
HTML, CSS, {\LaTeX}
\subsection{Banco de Dados}
PostgreSQL/PostGIS, SQLite/SpatiaLite, Microsoft SQL Server , Geodatabase/Geopackage
\subsection{Tecnologias Web}
Node.js, Geoserver
\subsection{Ferramentas Computacionais}
\subsubsection{Bibliotecas}
Qt, Boost, GEOS, GDAL, OGR, Weka, DotSpatial, SharpMap
\subsubsection{Análise e visualização de dados} 
Tableau, MS/Libre Office
\subsubsection{Outras tecnologias}
Git, Docker, rsync, cron, ssh, tmux, vim, pkg-config, rpm, aptitude, dialog 

% ------------------------------------------------------ % 

\section{Atuação Profissional}
\subsection{Fundo Verde}
\textbf{(2016 - Atual) - Analista de Banco de Dados Espaciais} \\ \\
Atua em projetos ligados ao monitoramento de vegetação costeira, estudo e implantação de tecnologias em energias renováveis, estudos de variações microclimáticas em ambientes urbanos e de tecnologias disruptivas ligadas ao transporte intra-urbano.

\subsection{AMS Kepler Engenharia de Sistemas}
\textbf{(2015 - 2015) - Analista de Geoprocessamento} \\ \\
Atuou em projetos junto ao Ministério de Meio Ambiente (MMA) para o mapeamento de grandes extensões de terra utilizando técnicas de classificação baseada em objetos (GEOBIA) em imagens de satélite de alta resolução espacial (RapidEye) utilizando o software eCognition.

\subsection{Laboratório ESPAÇO de Sensoriamento Remoto e Estudos Ambientais - IGEO/UFRJ}
\textbf{(2012 - Atual) - Pesquisador} \\ \\
Participação em pesquisa e desenvolvimento de atividades ligadas as áreas de Geoprocessamento e Sensoriamento Remoto. Atualmente participa do projeto Olho no Verde ligado ao INEA.

\subsection{Climate Policy Initiative}
\textbf{(2014 - 2014) - Research Assistant} \\ \\
Atuou em projetos envolvendo estudos de uso do solo no Brasil utilizando técnicas de análise espacial, SIG/Sensoriamento Remoto e automatização de processos utilizando métodos computacionais.

\subsection{Universidade do Estado do Rio de Janeiro - UERJ}
\textbf{(2012 - 2013) - Monitoria (Cartografia Temática)} \\
\textbf{(2011 - 2011) - Iniciação à docência} \\
\textbf{(2010 - 2011) - Iniciação científica}

% ------------------------------------------------------ % 

\section{Prêmios}
\textbf{2013 - Melhor trabalho na categoria Painéis no XVI SBSR, Instituto Nacional de Pesquisas Espaciais - INPE.}

\section{Idiomas}
\subsubsection{Inglês} 
Compreende Bem, Fala Razoavelmente, Lê Bem, Escreve Bem.
\subsubsection{Espanhol}
Compreende Razoavelmente, Fala Pouco, Lê Razoavelmente, Escreve Pouco.

\end{document}